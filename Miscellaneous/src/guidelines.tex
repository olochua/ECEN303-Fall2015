\documentclass[11pt]{article}

%%  Dimensions and URL
\usepackage[margin=1in]{geometry}
\usepackage{color,adjustbox,amsmath}

%%  Definitions
\renewcommand{\baselinestretch}{1.1}
\pagestyle{plain}

%%  Cases
\newif\ifsoln
\solnfalse


\begin{document}

\begin{center}
{\bfseries \LARGE Assignment Guidelines}
\end{center}

Random Signals \& Systems is a high enrollment course.
To streamline the grading process, students are required to adhere to the following guidelines while producing solutions to assignments.
In addition, you are expected to write neatly and legibly.
Your work reflects on your professionalism and diligence as future engineers.
Assignments that do not comply with these guidelines may be returned with a grade of zero.


\paragraph{Student Identity:} At the top-left corner of every assignment, provide the following information in the format shown below.

\noindent
\rule[1mm]{\linewidth}{0.5pt}
\begin{tabular}{ll}
\textbf{Name:} & \texttt{Full Name} \\
\textbf{NetID:} & \texttt{TAMU NetID} \\
\textbf{UIN:} & \texttt{TAMU UIN}
\end{tabular}


\paragraph{Assignment and Section Numbers:}
Clearly identify the assignment number and the course section at the top of every solution set.
\begin{center}
\begin{Large}
\textbf{Assignment 1 -- ECEN 303-502}
\end{Large}
\end{center}


\paragraph{Problem Differentiation:}
To distinguish the work associated with each problem, clearly outline the beginning of every question with a number and a colored line in the following manner.

\noindent
\textcolor{red}{\rule[1mm]{\linewidth}{1pt}}
\begin{Large}
\textcolor{red}{\textbf{1.}}
\end{Large}


\paragraph{Solution Highlight:}
When a question leads to a succinct answer, for example a number, highlight the answer in a colored box.

\noindent
\adjustbox{cframe=red 1pt 8pt 8pt}{\Large $e^{j \pi} + 1 = 0$}


\paragraph{Template:}
You will find a solution template below.


\newpage


\noindent
\begin{tabular}{ll}
\textbf{Name:} & \texttt{Jean-Francois Chamberland} \\
\textbf{NetID:} & \texttt{chmbrlnd} \\
\textbf{UIN:} & \texttt{271828182}
\end{tabular}

\vspace{5mm}

\begin{center}
\begin{Large}
\textbf{Assignment 1 -- ECEN 303-502}
\end{Large}
\end{center}

\vspace{5mm}

\noindent
\textcolor{red}{\rule[1mm]{\linewidth}{1pt}}
\begin{Large}
\textcolor{red}{\textbf{1.}}
\end{Large}
What is the probability of getting four kings in a poker hand?

\vspace{5mm}

There are $\binom{52}{5}$ ways to get a 5-card poker hand.
All these poker hands are equally likely when a deck is well shuffled.
Out of these possible outcomes, there are $12$ hands with four kings.
As such, the probability of getting four kings in a poker hand is
\begin{equation*}
\frac{48}{\binom{52}{5}}
= \frac{48 \cdot 47! \cdot 5!}{52!}
= \frac{5 \cdot 4 \cdot 3 \cdot 2 \cdot 1}{52 \cdot 51 \cdot 50 \cdot 49}
= \frac{1}{54 145} .
\end{equation*}

\noindent
\adjustbox{cframe=red 1pt 8pt 8pt}{\Large $\frac{1}{54 145}$}


\end{document}

